\documentclass[12pt,a4paper]{article}

% --- 中文支持 ---
\usepackage{xeCJK}           % XeLaTeX 中文支持
\setCJKmainfont{SimSun}      % 宋体为主字体(可改为 SimHei/NSimSun 等)
\setCJKmonofont{SimSun}      % 等宽字体
\setCJKfamilyfont{hei}{SimHei} % 黑体
\newcommand{\hei}[1]{{\CJKfamily{hei}#1}}

\usepackage{amsmath, amssymb}
\usepackage{geometry}
\usepackage{graphicx}
\usepackage{float}
\usepackage{hyperref}
\usepackage{booktabs}
\usepackage{multirow}

\geometry{left=2.5cm,right=2.5cm,top=2.5cm,bottom=2.5cm}

\title{\hei{基于光学 ToF 与被动麦克风阵列 AoA 的多模态测距与定位传感系统设计}}



\begin{document}

\maketitle
\tableofcontents
\newpage

\section{设计背景}
在常用测距测位传感系统中多为光学测量或声学测量,单一传感模态在复杂环境(如玻璃、低反射表面、噪声环境)下易失效。多模态融合可提升鲁棒性与精度。光学 ToF 模块提供高精度距离测量,但对透明或镜面物体失效;被动麦克风阵列通过声源方向估计(AoA)补充信息。结合视觉检测进一步增强系统能力。


\section{总体设计思路}
\subsection{目标}
设计并实现一个“以光学 ToF 为主、被动麦克风阵列 AoA 为辅”的多模态传感系统,用于室内近距(约0.2–3 m)物体/声源的定位。系统通过模态冗余与失效补偿提高在复杂表面(玻璃、低反射、暗色)及噪声环境下的鲁棒性。重点在传感器链路(物理→电信号→数字采样)、时序/校准与误差分析,算法采用可解释、计算量可控的方案。

\subsection{系统构成}
\begin{itemize}
    \item 主传感:光学 ToF 模块(主动,直接测距)
    \item 被动声学:MEMS 麦克风阵列 → AoA(方向)估计(GCC-PHAT)
    \item 主控平台:STM32F4/F7 MCU(实时采样 + DSP)
    \item 数据融合:置信度加权为主,卡尔曼滤波(KF/EKF)可选
\end{itemize}

\section{设计选择理由}
\begin{enumerate}
    \item 互补性:ToF 在多数材质/照明下精度高但对透明/镜面失效;被动声学可提供方向信息。
    \item 资源可控:现成 ToF 模块 + MEMS 麦克风阵列 + STM32 可完成硬件链路、信号处理与基础融合。
    \item 契合度高:覆盖传感器物理、模拟前端、采样时序、校准与误差分析。
    \item 可扩展性:可后续增加 EKF、更多麦克风阵列或视觉模块,形成多模态融合
\end{enumerate}

\section{物理原理概述}
\subsection{光学 ToF}
发送短脉冲光,测量往返时间 $\Delta t$:
\[
r_{\text{opt}} = \frac{c \cdot \Delta t}{2}
\]
其中 $c$ 为光速。模块输出距离和信号质量指标。
特点:精度高,响应快;对透明或低反射表面可能失效。

\subsection{单目视觉}
用于目标检测和角度估计。若已知目标尺寸,可粗略估计深度:
\[
d \approx \frac{f \cdot W}{p}
\]
$f$为焦距(像素),$W$为目标实际宽度,$p$为像素宽度。

\subsection{被动声学 TDOA / AoA}
\begin{itemize}
    \item 原理:声源发声,声波到达各麦克风的时间不同,测量通道间到达时间差$\Delta t$(TDOA),由阵列几何推算入射角(AoA)。
    \item 限制:只能测方向,不能直接测距;需要目标或环境发声。
\end{itemize}

\section{核心算法概念}
\subsection{GCC-PHAT TDOA}
\begin{enumerate}
    \item FFT 得到每通道频域表示 $X_k(f)$
    \item 计算互谱 $R_{ij}(f) = X_i(f) X_j^*(f)$
    \item PHAT 加权:
    \[
    R_{ij}^{PHAT}(f) = \frac{R_{ij}(f)}{|R_{ij}(f)| + \varepsilon}
    \]
    \item IFFT 得到互相关函数 $r_{ij}(\tau)$,峰值对应延迟 $\hat{\tau}_{ij}$
    \item 根据阵列几何求 AoA,结合多通道加权平均生成最终方向
\end{enumerate}

\subsection{视觉检测与 bearing}
\[
\theta_v = \arctan\left(\frac{u-c_x}{f_x}\right)
\]

\subsection{ToF 读取与质量指标}
ToF 输出距离 $r$ 与回波强度 $S$ 或信号质量 $Q$,映射为置信度 $C_{\text{opt}}$。

\subsection{融合策略}
角度融合:
\[
\theta_f = \frac{C_v \theta_v + C_a \theta_a}{C_v + C_a}
\]

距离选择:若 $C_{\text{opt}}$ 高,用 ToF 距离;否则使用视觉或其他策略。

线性 KF 或 EKF 对时间序列观测进行平滑/跟踪。



\section{关键物料及作用}
\begin{itemize}
    \item \textbf{STM32F4/F7 MCU}:采样、DSP运算、融合逻辑
    \item \textbf{光学 ToF 模块}:主动测距,提供置信度
    \item \textbf{单目摄像头}:目标检测、方向信息
    \item \textbf{MEMS 麦克风 ×4}:阵列采集声学信号,用于 AoA
    \item \textbf{前端/ADC}:信号放大滤波
    \item \textbf{电源/时钟/连接线}:稳定供电和时序
\end{itemize}

\section{AoA 原理}
对于两麦克风线性阵列:
\[
\Delta t = \frac{B \sin\theta}{c} \quad \Rightarrow \quad \theta = \arcsin\left(\frac{c \Delta t}{B}\right)
\]
\begin{itemize}
    \item 无法直接测距,只提供方向线
    \item 必须有声源(目标或环境音)
\end{itemize}

\section{TDOA / GCC-PHAT 参数考量}
\begin{itemize}
    \item 采样率 $f_s$:分辨率 $1/f_s$,可提升亚样本精度
    \item 窗长 $N$:平衡时域/频域分辨率
    \item 多径处理:选择直接到达峰
    \item 阵列基线 $B$:越长角度分辨率越高,但受物理限制
\end{itemize}

\section{置信度设计}
\begin{itemize}
    \item 光学 ToF 置信度 $C_{\text{opt}}$:回波强度/信号质量
    \item 视觉置信度 $C_v$:检测器置信度 + bbox 稳定性
    \item 声学置信度 $C_a$:互相关峰值/背景能量比 + 多通道一致性
\end{itemize}

\section{卡尔曼滤波(KF/EKF)}
\begin{itemize}
    \item 线性 KF:适用于线性状态与观测
    \item EKF:处理非线性观测(如 bearing, r)
    \item 实现难度中等,STM32F4 完全可用
\end{itemize}

\section{误差分析}
\subsection{误差来源}
\begin{itemize}
    \item ToF:时间测量噪声、表面反射、背景光
    \item 视觉:相机内参、像素量化、目标检测错误
    \item AoA:采样/量化误差、阵列基线测量误差、多径、声速偏差
\end{itemize}

\subsection{误差传播公式}
若 $x = d \cos \theta, y = d \sin \theta$:
\[
\sigma_x^2 = (\cos\theta)^2 \sigma_d^2 + (d\sin\theta)^2 \sigma_\theta^2, \quad
\sigma_y^2 = (\sin\theta)^2 \sigma_d^2 + (d\cos\theta)^2 \sigma_\theta^2
\]

\section{校准流程}
\begin{enumerate}
    \item 相机内参标定
    \item ToF 偏置校准
    \item 麦克风阵列时延和增益校准
    \item 传感器外参标定
    \item 温度校正(声速)
    \item 测试与验证
\end{enumerate}





\end{document}
